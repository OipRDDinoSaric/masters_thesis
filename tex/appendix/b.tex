\chapter{Developed bootloader command reference} % Grammarly OK
\label{appendix_cmd_ref_bootloader}

Important notices:

\begin{itemize}
 \item Every execute of a command must end with \textbackslash r \textbackslash n

 \item Commands are case insensitive
 
 \item On error bootloader returns "ERROR:<Explanation of error>"
 
 \item Optional parameters are surrounded with [] e.g. [example]

\end{itemize}

\noindent List of all commands:

\begin{itemize}

    \item \nameref{bl_cmd:version}
    \item \nameref{bl_cmd:help}
    \item \nameref{bl_cmd:reset}
    \item \nameref{bl_cmd:cid}
    \item \nameref{bl_cmd:get-rdp-level}
    \item \nameref{bl_cmd:jump-to}
    \item \nameref{bl_cmd:flash-erase}
    \item \nameref{bl_cmd:flash-write}
    \item \nameref{bl_cmd:mem-read}
    \item \nameref{bl_cmd:update-act} 
    \item \nameref{bl_cmd:update-new}
    \item \nameref{bl_cmd:en-write-prot}
    \item \nameref{bl_cmd:dis-write-prot}
    \item \nameref{bl_cmd:get-write-prot}
    \item \nameref{bl_cmd:exit}

\end{itemize}



\section{version - Gets a version of the bootloader.}
\label{bl_cmd:version}

Parameters:
\begin{lstlisting}
 - None
Execute command: 

    > version  
    
Response: 

    v1.0  
\end{lstlisting}
    
    
\section{help - Makes life easier.}
\label{bl_cmd:help}

\begin{lstlisting}
Parameters:

 -  None

Execute command: 

    > help  
    
Response: 

    <List of all available commands and examples>
\end{lstlisting}

\section{reset - Resets the microcontroller.}
\label{bl_cmd:reset}

\begin{lstlisting}
Parameters:

 - None

Execute command: 

    > reset
    
Response: 

    OK
\end{lstlisting}


\section{cid - Gets chip identification number.}
\label{bl_cmd:cid}

\begin{lstlisting}
Parameters:

 - None

Execute command: 

    > cid  
    
Response: 

    0x413
\end{lstlisting}

\section{get-rdp-level - Gets read protection \texorpdfstring{\protect\cite[p.~93]{stm32f407_ref_man}}{}}
\label{bl_cmd:get-rdp-level}

\begin{lstlisting}
Parameters:

  - None

Execute command: 

    > get-rdp-level  
    
Response: 

    level 0
\end{lstlisting}
    
\section{jump-to - Jumps to a requested address.}
\label{bl_cmd:jump-to}

\begin{lstlisting}
Parameters:

- addr - Address to jump to in hex format (e.g. 0x12345678), 0x can be omitted

Execute command: 

    > jump-to addr=0x87654321 
     
Response: 

    OK
\end{lstlisting}
     
\section{flash-erase - Erases flash memory.}
\label{bl_cmd:flash-erase}

\begin{lstlisting}
Parameters:

- type - Defines type of flash erase. "mass" erases all sectors, "sector" erases only selected sectors
    
- sector - First sector to erase. Bootloader is on sectors 0, 1 and 2. Not needed with mass erase
    
- count - Number of sectors to erase. Not needed with mass erase

Execute command: 

    > flash-erase sector=3 type=sector count=4 
     
Response: 

    OK
\end{lstlisting}

\section{flash-write - Writes to flash memory.}
\label{bl_cmd:flash-write}

\begin{lstlisting}
Parameters:

 - start - Starting address in hex format (e.g. 0x12345678), 0x can be omitted
     
 - count - Number of bytes to write, without checksum. Chunk size: 5120
 
 - [cksum] - Defines the checksum to use. If not present no checksum is assumed. WARNING: Even if checksum is wrong data will be written into flash memory!
 
      - "sha256" - Gives best protection (32 bytes), slowest, uses software implementation
           
      - "crc32" - Medium protection (4 bytes), fast, uses hardware implementation.

      - "no" - No protection, fastest

Note:

  When using crc-32 checksum sent data has to be divisible by 4

Execute command: 

    > flash-write start=0x87654321 count=64 cksum=crc32  
    
Response: 

    chunks:1

    chunk:0|length:64|address:0x87654321

    ready
    
Send bytes:

    <64 bytes>
    
Response:

    chunk OK

    checksum|length:4

    ready
 
Send checksum:
     
     <4 bytes>
     
Response:
 
    OK
\end{lstlisting}

\section{mem-read - Read bytes from memory.}
\label{bl_cmd:mem-read}

\begin{lstlisting}
Parameters:

- start - Starting address in hex format (e.g. 0x12345678), 0x can be omitted
     
- count - Number of bytes to read

Execute command: 

    > mem-read start=0x87654321 count=3  
    
Response: 

    <3 bytes starting from the address 0x87654321>
    
Note:
- Entering invalid read address crashes the program and reboot is required. 
\end{lstlisting}

\section{update-act - Updates active application from new application memory area.}
\label{bl_cmd:update-act}

\begin{lstlisting}
Parameters:
- [force] - Forces update even if not needed

   - "true" - Force the update
                
   - "false" - Don't force the update


Execute command: 

    > update-act force=true
    
Response: 

    No update needed for user application
    Updating user application
    OK
\end{lstlisting}
    

\section{update-new - Updates new application.}
\label{bl_cmd:update-new}

\begin{lstlisting}
Parameters:

 - count - Number of bytes to write, without checksum. Chunk size: 5120
 
 - type - Type of application coding
       
      - "bin" - Binary format (.bin)
                
      - "hex" - Intel hex format (.hex)
      
      - "srec" - Motorola S-record format (.srec)
 
 - [cksum] - Defines the checksum to use. If not present no checksum is assumed. WARNING: Even if checksum is wrong data will be written into flash memory!
 
      - "sha256" - Gives best protection (32 bytes), slowest, uses software implementation
           
      - "crc32" - Medium protection (4 bytes), fast, uses hardware implementation.

      - "no" - No protection, fastest


Execute command: 

    > update-new count=4 type=bin cksum=sha256
    
Response: 

    chunks:1

    chunk:0|length:4|address:0x08080000

    ready
    
Send bytes:

    <4 bytes>
    
Response:

    chunk OK

    checksum|length:32

    ready
 
Send checksum:
     
     <32 bytes>
     
Response:
 
    OK
\end{lstlisting}

\section{en-write-prot - Enables write protection per sector.}
\label{bl_cmd:en-write-prot}

\begin{lstlisting}
Parameters:

- mask - Mask in hex form for sectors where LSB corresponds to sector 0

Execute command: 

    > en-write-prot mask=0xFF0
    
Response: 

    OK
\end{lstlisting}
    
\section{dis-write-prot - Disables write protection per sector.}
\label{bl_cmd:dis-write-prot}

\begin{lstlisting}
Parameters:

- mask - Mask in hex form for sectors where LSB corresponds to sector 0

Execute command: 

    > dis-write-prot mask=0xFF0
    
Response: 

    OK
\end{lstlisting}

\section{get-write-prot - Returns bit array of sector write protection.}
\label{bl_cmd:get-write-prot}Parameters:

\begin{lstlisting}
- None

Execute command: 

    > get-write-prot
    
Response: 

    0b100000000010
\end{lstlisting}

\section{exit - Exits the bootloader and starts the user application.}
\label{bl_cmd:exit}

\begin{lstlisting}
Parameters:

- None

Execute command: 

    > exit  
    
Response: 

    Exiting
\end{lstlisting}

