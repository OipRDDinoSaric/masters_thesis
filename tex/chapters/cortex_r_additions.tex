\chapter{ARM Cortex R comparison with ARM cortex M}
\label{cortex_r_additions}

\section{ARM Cortex R introduction}

The ARM Cortex R is a family of 32-bit RISC processors. The cores are optimized for hard real-time and safety-critical applications. Cortex R family consists of: ARM Cortex-R4(F), ARM Cortex-R5(F), ARM Cortex-R7(F), ARM Cortex-R8(F and ARM Cortex-R52(F).

The Cortex-R is suitable for use in computer-controlled systems where very low latency and/or a high level of safety is required. An example of a hard real-time, safety critical application would be a modern electronic braking system in an automobile. The system not only needs to be fast and responsive to a plethora of sensor data input, but is also responsible for human safety. A failure of such a system could lead to severe injury or loss of life. Other examples of hard real-time and safety critical applications include medical devices, electrionic control units (ECU), robotics etc.

\section{ARM Cortex M introduction}

The ARM Cortex M is also a family of 32-bit RISC processors. The cores are optimized for low-cost and energy-efficient microcontrollers. Theese cores are being used in tens of billions of consumer devices. The processors are intended for deeply embedded applications that require fast interrupt response features. \citep{cortex_m4_reference} The family consists of: Cortex-M0, Cortex-M0+, Cortex-M1, Cortex-M3, Cortex-M4, Cortex-M7, Cortex-M23, Cortex-M33, Cortex-M35P and Cortex-M55.


\section{ARM Cortex-R5 processor lockstep}

\subsection{Dual core lockstep}

\todoi{dual code lockstep, mention delayed lockstep CPU compare}

\subsection{Triple core lockstep}
\todoi{dual code lockstep}

\section{Sto jos dodaje Cortex R}

\todoi{data cache, instruction cache, atcm, btcm, preformance monitor unit, interrupt controller connected though porr, more pipeline stages}

