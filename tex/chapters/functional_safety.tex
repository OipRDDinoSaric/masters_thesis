\chapter{Functional safety in embedded systems}
\label{functional_safety}

\section{IEC standards}

A variety of standards exist for specific or general markets, to facilitate the compliance of the systems to the functional safety principles. The standard IEC 61508 gives methods on how to apply, design, deploy and maintain automatic protection systems called safety-related systems.
Along with the more general standard, industry specific standards exist:
\begin{itemize}

    \item ISO 26262 for automotive passenger vehicles,
    \item IEC 61511 for the process industry and associated instrumentation,
    \item EN 50128 for software development of railway applications,
    \item IEC 61513 for nuclear power plants,
    \item IEC 62061 and ISO 13849 for machinery electrical control systems,
    \item IEC 62304 for medical systems and
    \item IEC 60730 for white goods.

\end{itemize}

Mentioned standards provide guidelines to assess rick and assign safety goals for safety-related systems of various industries. They provide frameworks for quantitative analysis of random failure rates and effectiveness of diagnostics to detect them. They also provide guidelines for maintenance of the safety-related systems after the deployment.

IEC 61508 is a basic functional safety standard applicable to all kinds of industry. It defines functional safety as: “part of the overall safety relating to the EUC (Equipment Under Control) and the EUC control system which depends on the correct functioning of the Electrical/Electronic/Programmable Electronic Safety-related Systems (E/E/PE) safety-related systems, other technology safety-related systems and external risk reduction facilities.” The fundamental concept is that any safety-related system must work correctly or fail in a predictable (safe) way. Functional safety relies only on active systems, and safety measures that rely on passive systems are not functional safety. \citep{func_safety_explained}

\section{Terminology}


\todoi{Functional safety terminology, SIL, SIF, SIS...}

\section{Certification process}

\todoi{Procjena etc.}

\section{Embedded processors redundancy - lockstep}

Consider a system with one processing unit. Such system is obviously non-redundant from the
aspect of instruction execution. Lockstep is redundancy mechanism for increasing the system reliability
by introducing at least one redundant processing unit. Redundant processing unit replicates the behavior of
the original processing unit. Depending on number of processing units lockstep can provide fault-detection, if there are less than three processing units in a system, or both fault-detection and
fault-tolerance with more than two processing units in a system. \citep{ipavic_lockstep}

Locksteped processors can get upto SIL3, but not SIL4 as a use of on-chip redundancy is limited by IEC 61508 standard to SIL3. Therefore, although it may seem, lockstep is certainly not panacea in all safety systems and certain amount of work is needed to justify its usage. Nevertheless, for all safety integrity levels, but SIL4 lockstep is often used extensively (e.g. automotive applications). \citep{ipavic_lockstep}