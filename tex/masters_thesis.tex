\documentclass[utf8, diplomski, english, numeric]{fer}
\usepackage{booktabs}
\usepackage{caption}
\usepackage{subcaption}
\usepackage{graphicx}
\usepackage{grffile}
\usepackage{listings}
\usepackage[table,xcdraw]{xcolor}
\usepackage{multirow}
\usepackage{dirtree}
\usepackage{float}
\usepackage[hidelinks]{hyperref}
\usepackage[utf8]{inputenc}
\usepackage[english]{babel}
\usepackage[fixlanguage]{babelbib}
\usepackage[outputdir=build]{minted}
\usepackage{todonotes}
\usepackage[title]{appendix}

\lstset{
basicstyle=\small\ttfamily,
columns=flexible,
breaklines=true
}

\newcommand{\todoi}[1]{\todo[inline]{#1}}

\newcommand{\code}[1]{\texttt{#1}}

\def\changemargin#1#2{\list{}{\rightmargin#2\leftmargin#1}\item[]}
\let\endchangemargin=\endlist 

\begin{document}

\thesisnumber{1981}

\title{Software and Hardware Architecture for Redundant Embedded Systems}

\author{Dino Šarić}

\maketitle

\izvornik
%\includepdf[pages={1-}]{izvornik.pdf}

\zahvala{Thanks to my parents for supporting me financially and giving me an opportunity of studying away from home. Thanks to my aunt Dijana and my friends for supporting me emotionally. 

Thanks to my mentor izv. prof. dr. sc. Hrvoje Džapo and his assistant Ivan Pavić mag. ing. for a friendly, fast and diligent approach during the studies and the writing of the thesis. Thanks to all the helpful current and former students from the college forums. Finally, thanks to the YouTubers that made understanding the college curriculum much easier.} % Gramarly OK

\tableofcontents
\listoffigures
\listoftables

\chapter*{Introduction}
Uvod u rad.
\chapter{Functional safety in embedded systems}
\label{functional_safety}

\section{IEC standards}

A variety of standards exist for specific or general markets, to facilitate the compliance of the systems to the functional safety principles. The standard IEC 61508 gives methods on how to apply, design, deploy and maintain automatic protection systems called safety-related systems.
Along with the more general standard, industry specific standards exist:
\begin{itemize}

    \item ISO 26262 for automotive passenger vehicles,
    \item IEC 61511 for the process industry and associated instrumentation,
    \item EN 50128 for software development of railway applications,
    \item IEC 61513 for nuclear power plants,
    \item IEC 62061 and ISO 13849 for machinery electrical control systems,
    \item IEC 62304 for medical systems and
    \item IEC 60730 for white goods.

\end{itemize}

Mentioned standards provide guidelines to assess rick and assign safety goals for safety-related systems of various industries. They provide frameworks for quantitative analysis of random failure rates and effectiveness of diagnostics to detect them. They also provide guidelines for maintenance of the safety-related systems after the deployment.

IEC 61508 is a basic functional safety standard applicable to all kinds of industry. It defines functional safety as: “part of the overall safety relating to the EUC (Equipment Under Control) and the EUC control system which depends on the correct functioning of the Electrical/Electronic/Programmable Electronic Safety-related Systems (E/E/PE) safety-related systems, other technology safety-related systems and external risk reduction facilities.” The fundamental concept is that any safety-related system must work correctly or fail in a predictable (safe) way. Functional safety relies only on active systems, and safety measures that rely on passive systems are not functional safety. \citep{func_safety_explained}

\section{Terminology}


\todoi{Functional safety terminology, SIL, SIF, SIS...}

\section{Certification process}

\todoi{Procjena etc.}

\section{Embedded processors redundancy - lockstep}

Consider a system with one processing unit. Such system is obviously non-redundant from the
aspect of instruction execution. Lockstep is redundancy mechanism for increasing the system reliability
by introducing at least one redundant processing unit. Redundant processing unit replicates the behavior of
the original processing unit. Depending on number of processing units lockstep can provide fault-detection, if there are less than three processing units in a system, or both fault-detection and
fault-tolerance with more than two processing units in a system. \citep{ipavic_lockstep}

Locksteped processors can get upto SIL3, but not SIL4 as a use of on-chip redundancy is limited by IEC 61508 standard to SIL3. Therefore, although it may seem, lockstep is certainly not panacea in all safety systems and certain amount of work is needed to justify its usage. Nevertheless, for all safety integrity levels, but SIL4 lockstep is often used extensively (e.g. automotive applications). \citep{ipavic_lockstep}
\chapter{ARM Cortex-R functional safety additions over ARM Cortex-M}
\label{cortex_r_additions}

\section{ARM Cortex-R introduction}

The ARM Cortex-R is a family of 32-bit RISC processors. The cores are optimized for hard real-time and safety-critical applications. Cortex-R family consists of ARM Cortex-R4(F), ARM Cortex-R5(F), ARM Cortex-R7(F), ARM Cortex-R8(F and ARM Cortex-R52(F).

The Cortex-R is suitable for use in computer-controlled systems where very low latency and/or a high level of safety is required. An example of a hard real-time, safety-critical application would be a modern electronic braking system in an automobile. The system not only needs to be fast and responsive to a plethora of sensor data input but is also responsible for human safety. A failure of such a system could lead to severe injury or loss of life. Other examples of hard real-time and safety-critical applications include medical devices, electronic control units (ECU), robotics, etc.

\section{ARM Cortex-M introduction}

The ARM Cortex-M is also a family of 32-bit RISC processors. The cores are optimized for low-cost and energy-efficient microcontrollers. These cores are being used in tens of billions of consumer devices. The processors are intended for deeply embedded applications that require fast interrupt response features \citep{cortex_m4_reference}. The family consists of Cortex-M0, Cortex-M0+, Cortex-M1, Cortex-M3, Cortex-M4, Cortex-M7, Cortex-M23, Cortex-M33, Cortex-M35P and Cortex-M55.


\section{ARM Cortex-R processor lockstep}

\subsection{Dual core lockstep}

A Cortex-R5 processor group has four configurations, as described in \citep{cortex_r5_reference_manual}:
\begin{itemize}

    \item single CPU,
    \item twin CPU,
    \item redundant CPU and
    \item split/lock configuration.
    
\end{itemize}

Twin CPU configuration includes two individual and decoupled CPUs. Each CPU has its own
cache RAMs, debug logic and bus interfaces to the rest of the SoC. It
offers higher performance than a standard single CPU configuration. 

In redundant CPU configuration (lockstep), there is a functional CPU and a second redundant copy of the majority of the CPU logic. The redundant logic is driven by the same inputs as the functional logic.  In particular, the redundant CPU logic shares the same cache RAMs as the functional CPU. Therefore only one set of cache RAMs is required. The redundant logic
operates in lock-step with the CPU but does not directly affect the processor behavior in any way \citep{cortex_r5_reference_manual}. The CPU outputs to the cache RAMs are driven \textbf{exclusively} by the functional CPU. The comparison logic for comparing the outputs of the redundant logic and the functional logic can detect a single fault that occurs in either set of logic. ARM provides example comparison logic, but the developer can change it during the implementation. 

Split/lock configuration is a combination of two previously mentioned configurations. This mode includes two processors. If a processor is in split mode the CPUs work in twin CPU configuration, and if the processor is in lock mode it is in a redundant configuration.

\begin{figure}[H]

      \centering
      \includegraphics[width=1\linewidth]{images/split_lock_configuration.png}
      \caption{Split/lock configuration \citep{cortex_r8_reference_manual}}
      \label{fig:split_lock_configuration}
    
\end{figure}

\subsection{Triple core lockstep}

ARM triple-core lockstep architecture (TCLS) builds upon the industry success of the ARM Cortex-R5 dual-core lock-step (DCLS). The TCLS architecture adds a third redundant CPU unit to the DCLS Cortex-R5 system to achieve fail functional capabilities and hence increase the availability of the system \citep{TCLS_cortex_r}.

Cores in triple lockstep have shared data and instruction cache, but each Cortex-R5 has its own clock tree. \autoref{fig:tcls_architecture} shows the system level solution to mitigate soft errors occurring in the redundant CPUs. On the right side of the figure, the TCLS assist unit is visualized. TCLS assist unit supports the lockstep functioning of the CPUs and handles the error recovery process. The unit consists of a majority voter, error detection logic and synchronization logic.


\begin{figure}[H]

      \centering
      \includegraphics[width=0.7\linewidth]{images/tcls_architecture.png}
      \caption{ARM triple core lockstep of Cortex-R5 \citep{TCLS_cortex_r}}
      \label{fig:tcls_architecture}
    
\end{figure}

At every clock cycle, the instructions to execute are read
from the shared instruction cache or TCM (tightly coupled memory) and distributed to
the triplicated CPUs. The outputs from the CPUs are majority voted and
forwarded to the shared data cache, TCM, and I/O ports.
Simultaneously, the Error Detection logic checks if there is any
mismatch in the outputs delivered by the three CPUs. If there is
a mismatch, all CPUs are interrupted and the Error Detection
logic identifies whether it is a correctable error (i.e., only one
of the CPUs delivers a different set of outputs) or an
uncorrectable one (i.e., all CPUs deliver different outputs). If
the error is correctable, the TCLS passes the control to the
resynchronization logic to correct the architectural state of the
erroneous CPU, that is, to resynchronize all the CPUs. Note
here that the Majority Voter acts as an error propagation
boundary, preventing correctable errors from propagating to
memories and I/O ports. In the highly unlikely case that the
error is uncorrectable, the TCLS transitions to a fail-safe
operation state \citep{TCLS_cortex_r}.

As mentioned in the last paragraph, the majority voter is in the critical path of the system, but the error detection logic is out of the critical path and is pipelined to increase performance.

\autoref{fig:tcls_resynchronization} show the flow diagram of the resynchronization logic. Upon a correctable error is detected, the resynchronization logic can immediately trigger the CPU resynchronization process. This action prevents the interruption of critical real-time tasks. It should be noted that the system can continue working on two remaining CPUs, which are in a functionally correct state. The third CPU is recovered from the two functioning CPUs. Unlike the dual-core lockstep, the recovery of the triple-core lockstep is automatic and transparent to the software \citep{TCLS_cortex_r}.

\begin{figure}[H]

      \centering
      \includegraphics[width=0.9\linewidth]{images/tcls_resynchronization.png}
      \caption{TCLS resynchronization finite state machine \citep{TCLS_cortex_r}}
      \label{fig:tcls_resynchronization}
    
\end{figure}

\section{RAM error correction}

In microcontrollers, stray radiation and other effects can cause the data stored in RAM (random access memory) to be corrupted (bit flip).  The tightly coupled memories (TCMs) and caches on a Cortex-R processor can be configured to detect and correct errors that can occur in the RAMs. Extra, redundant data (checksum) is computed by the processor and stored in RAMs alongside the real data. When the processor reads the data from the RAM, it checks the redundant data matches the real data and can either signal an error, or attempt to correct the data.

Cortex-R5 allows the usage of:
\begin{itemize}

    \item parity,
    \item 64-bit ECC or
    \item 32-bit ECC \citep{cortex_r5_reference_manual}.
\end{itemize}

With parity, an error can be detected, but it can't be corrected. Using either 64-bit or 32-bit ECC (error checking and correction) detects up to two errors in a data chunk (either 64-bit or 32-bit) and correct any single error in the data chunk. Parity adds one redundant bit for every byte. 64-bit ECC adds one byte per 8-byte data chunk. 32-bit ECC adds seven bits per 4 bytes of the data chunk. 

Cortex-R8 uses 32-bit, 34-bit or 64-bit ECC of variable data chunks to protect its RAMs \citep{cortex_r8_reference_manual}.

\section{Interrupt handling}

According to \citep{cortex_r5_reference_manual}, Interrupt handling in the processor is compatible with previous ARM architectures, but has
several additional features to improve interrupt performance for real-time applications.

ARM Cortex-R5 connects the vectored interrupt controller (VIC) over a port, unlike ARM Cortex-M series microcontrollers that only have VIC close to the core. Having a VIC over the port provides faster interrupt entry, but one can disable it for compatibility with earlier interrupt controllers \citep{cortex_r5_reference_manual}. ARM Cortex-R5 doesn't allow tail chaining of the interrupts, nor the handling of late-arriving interrupts, unlike ARM Cortex-M series. 

Both ARM Cortex-R and Cortex-M series microcontrollers can abandon assembler instructions LDM and STM to lower the interrupt's latency. While Cortex-R microcontrollers will rerun the whole command upon the return from the interrupt, the ARM Cortex-M microcontrollers will just continue where they left of.

 

\section{CPU Compare Module for Cortex-R5F by Texas Instruments}

CPU compare module (CCM) detects run-time faults in devices like the CPU and VIM (vectored interrupt controller module) and forwards them to the next step in the error handling of the microcontroller i.e. to ESM (error signaling module). Alongside the run-time fault testing, CCM incorporates a self-test capability to allow for boot time checking of hardware faults within the CCM itself \citep{TMS570LS31x21x_manual}.

The main features of the CCM are:
\begin{itemize}

    \item{run-time detection of faults,}
    \item{self-test capability and}
    \item{error forcing capability.}

\end{itemize}

There are four modes of diagnostics for CPU/VIM output compare.
\begin{itemize}

    \item active compare lockstep,
    \item self-test,
    \item error forcing and
    \item self-test error forcing mode.

\end{itemize}

Active compare lockstep mode is defaulted to on start-up. The bus output signals of both CPU and VIMs are compared. List of all compared output signals is available in \citep[p. 500]{TMS570LS31x21x_manual}.

In self-test mode, test patterns are automatically generated by the CCM-R4F to
determine its correct functionality. In case of error detection that indicates a
hardware fault on the module itself, a “CCM-R4F - self-test” flag will be raised to
the ESM. After the completion or termination of the self-test, if no error occurred,
the self-test complete flag is set to notify the system to proceed appropriately. The compare match test and compare mismatch test are generated to ensure proper functioning. In compare match test an identical vector is applied to both input ports at the same time expecting a compare match. The compare mismatch test is similar to the compare match test, but one of the input vectors has one bit flipped. On the output, compare mismatch is expected.

Error forcing mode ensures that an error on the CCM's input is detected. In the case when the error from error forcing mode is not detected there is a hardware failure present. The difference between the compare mismatch test and error forcing mode is that error forcing mode ensures that compare error output signal asserts. This mode lasts for one CPU cycle.

\chapter{FreeRTOS kernel}
\label{freertos_kernel}

\section{Introduction}

FreeRTOS is a real-time operating system kernel (RTOS) for embedded devices. It has been ported to 35 microcontroller pletforms and has MIT open source licence hence the free in the name. \citep{freertos_licence}

Operating system (OS) is designed to be small and simple. The kernel itself has only three C files. It is written mostly in C with some assembly code included for scheduler routines.

FreeRTOS is ideally suited to deeply embedded real-time applications that use
microcontrollers or small microprocessors. This type of application normally includes a mix of
both hard and soft real-time requirements. \citep{freertos_mastering}

Soft real-time requirements are those that state a time deadline—but breaching the deadline
would not render the system useless. For example, responding to keystrokes too slowly might
make a system seem annoyingly unresponsive without actually making it unusable. \citep{freertos_mastering}

Hard real-time requirements are those that state a time deadline—and breaching the deadline
would result in absolute failure of the system. For example, a driver’s airbag has the potential
to do more harm than good if it responded to crash sensor inputs too slowly. \citep{freertos_mastering}

\noindent FreeRTOS features:

\begin{itemize}
    
    \item Pre-emptive or co-operative operation
    \item Very flexible task priority assignment
    \item Flexible, fast and light weight task notification mechanism
    \item Queues
    \item Binary and counting semaphores
    \item Mutexes
    \item Recursive mutexes
    \item Software timers
    \item Event groups
    \item Tick hook functions
    \item Idle hook functions
    \item Stack overflow checking
    \item Trace recording
    \item Task run-time statistics gathering
    \item Optional comemercial licensing and support
    \item Full interrupt nesting model (on some architectures)
    \item Tick-less capability for extreme low power applications
    \item Software manages interrrupt stack when appropriate (could save RAM space)
    
\end{itemize}

\noindent FreeRTOS file structure is shown in \autoref{fig:freertos_structure}.

\begin{figure}[H]
\dirtree{%
.1 Source.
.2 include.
.3 FreeRTOS.h.
.3 list.h - Primary structure used inside the kernel.
.3 message\textunderscore buffer.h.
.3 portable.h.
.3 projdefs.h.
.3 queue.h.
.3 semphr.h - Semaphores.
.3 stream\textunderscore buffer.h.
.3 task.h.
.3 timers.h - Software timers. 
.2 portable.
.3 [compiler] e.g{. GCC}.
.4 [architecture] e.g{. ARM\textunderscore CM4F}.
.5 port.c - Architecture and compiler specific.
.5 portmacro.c.
.3 MemMang.
.4 heap\textunderscore X.c - X is a heap number used.
.2 croutine.c.
.2 event\textunderscore groups.c.
.2 list.c.
.2 queue.c.
.2 stream\textunderscore buffer.c.
.2 tasks.c - Tasks and scheduler implementation.
.2 timers.c.
}
\caption{FreeRTOS file structure}
\label{fig:freertos_structure}
\end{figure}

\section{Inner working of the tasks}

\todoi{Inner working of tasks}


\section{Inner working of the timers}

\todoi{Inner working of timers}

\chapter{FreeRTOS redundancy API} % Grammarly OK
\label{freertos_modification}

\section{Timed tasks addition}

\subsection{Introduction}

Atop of all FreeRTOS functionality ability to measure task's run time and its total time (running + other states) is not provided. Such feature is extremely useful for hard real-time embedded systems. In such systems breaching the deadline means the failure of the system.

Adding the aforementioned timers to the tasks means the ability to detect if the task ran for too long or it didn't get enough processor time. When such an event has occurred it can be properly handled. For example, for an embedded system that is periodically polling the speed of rotation of an engine if it isn't polled in time, it can raise an alarm to force the reading. \autoref{appendix_cmd_ref_freertos} contains the command reference for the timed tasks.

\subsection{Architecture}


When timed tasks are created two software timers are created tied to it. One is called overrun timer and is used to detect when the task executes for longer than allocated. Second, the overflow timer is used to detect if the task is running properly asynchronously i.e. timer doesn't stop ticking even if the timed task is inactive.  

The overflow timer is started when the timed task is switched to the first time. It is started from the context switch. Code section is shown in \autoref{fig:freertos_overflow_start}. Function \code{prvStartOverflowTimer} checks if the timer is started and if it is not it starts the timer otherwise it does not do anything.


\begin{figure}[H]
\begin{changemargin}{1cm}{1cm}
\begin{lstlisting}[escapeinside={(*}{*)}, numbers=left, numberstyle=\tiny, stepnumber=1, language=c]
/* Check for stack overflow, if configured. */
taskCHECK_FOR_STACK_OVERFLOW();

/* Select a new task to run using either the generic C or port
   optimised asm code. */
taskSELECT_HIGHEST_PRIORITY_TASK();


#if( INCLUDE_xTaskCreateTimed == 1 )
{
   BaseType_t xHigherPriorityTaskWoken = pdFALSE;

   prvStartOverflowTimer( pxCurrentTCB, xIsSwitchContextFromISR, &xHigherPriorityTaskWoken );

   if(xHigherPriorityTaskWoken != pdFALSE)
   {
        /* Select a new task to run using either the generic C or port
           optimised asm code.
           Highest priority should be the timer daemon. */
        taskSELECT_HIGHEST_PRIORITY_TASK();
   }
}
#endif /* INCLUDE_xTaskCreateTimed == 1 */
\end{lstlisting}  
\end{changemargin}
\caption{Starting of the overflow timer from the file task.c}
\label{fig:freertos_overflow_start}
\end{figure}

The overrun timer is only a variable in the task control block. It is incremented every tick while the timed task is running. When the overrun timer is about to expire (1 tick before) a software timer is started. Next tick, the software timer triggers the overrun callback. Code for incrementing (\autoref{fig:freertos_overrun_increment}) is called from the function xTaskIncrementTick which itself is called from the tick interrupt. Overrun timer (of period 1 tick) is started from the context switch function, with code shown in \autoref{fig:freertos_overrun_start}.

\begin{figure}[H]
\begin{changemargin}{1cm}{1cm}
\begin{lstlisting}[escapeinside={(*}{*)}, numbers=left, numberstyle=\tiny, stepnumber=1, language=c]
#if INCLUDE_xTaskCreateTimed == 1
    BaseType_t prvIncrementOverrunTick(void)
    {
        BaseType_t xReturn = pdFALSE;
        if( pxCurrentTCB->xOverrunTimer != NULL )
        {
            pxCurrentTCB->xOverrunTicks++;
            if(pxCurrentTCB->xOverrunTicks >= ( pxCurrentTCB->xOverrunTicksMax - 1 ) )
            {
                xReturn = pdTRUE;
            }
        }
        return xReturn;
    }
#endif
\end{lstlisting}  
\end{changemargin}

\caption{Incrementing of the overrun timer from the file task.c}
\label{fig:freertos_overrun_increment}
    
\end{figure}

\begin{figure}[H]
\begin{changemargin}{1cm}{1cm}
\begin{lstlisting}[escapeinside={(*}{*)}, numbers=left, numberstyle=\tiny, stepnumber=1, language=c]
#if( INCLUDE_xTaskCreateTimed == 1 )
{
    if( pxCurrentTCB->xOverrunTimer != NULL )
    {
        if(pxCurrentTCB->xOverrunTicks >= ( pxCurrentTCB->xOverrunTicksMax - 1 ) )
        {
            /* Starts the timer with the period of 1. */
            xTimerResetFromISR( pxCurrentTCB->xOverrunTimer, NULL );
            pxCurrentTCB->xOverrunTicks = 0;
        }
    }
}
#endif /* INCLUDE_xTaskCreateTimed == 1 */
\end{lstlisting}  
\end{changemargin}
\caption{Starting of the overrun timer from the file task.c}
\label{fig:freertos_overrun_start}
    
\end{figure}


 The next three figures showcase how timed tasks work. \autoref{fig:timed_example_overrun} shows when is overrun timer triggered. Similarly, \autoref{fig:timed_example_overflow} showcases how the overflow timer works. Finally, \autoref{fig:timed_example_reset} shows how resetting the timed task suppresses the timeout of overflow and overrun timers.

\begin{figure}[H]

      \centering
      \includegraphics[width=\linewidth]{images/timed_example_overrun.png}
      \caption{Timed task with overrun timeout of 30 ms}
      \label{fig:timed_example_overrun}
    
\end{figure}

\begin{figure}[H]

      \centering
      \includegraphics[width=\linewidth]{images/timed_example_overflow.png}
      \caption{Timed task with overflow timeout of 20 ms}
      \label{fig:timed_example_overflow}
    
\end{figure}

\begin{figure}[H]

      \centering
      \includegraphics[width=\linewidth]{images/timed_example_reset.png}
      \caption{Timed task with both timers that resets in time}
      \label{fig:timed_example_reset}
    
\end{figure}

\code{vTaskDelete} function is changed so that deleting tasks also deletes their timers.

\subsection{Limitiations}

The static creation of the function is not available.

Timer callback functions are called by the timer daemon and its priority determines when the callback will be called. It is recommended that the timer daemon has the highest priority.

\section{Replicated tasks}

\subsection{Introduction}

Redundancy is a common term with safety hardware, but the redundancy can be achieved with the software. As is demonstrated with the replicated tasks. As the name suggests, when replicated tasks are created they make more parallel instances and they compare outputs of each other to assure no errors happened. 

Hardware redundancy has the advantage of detecting the fault as early as possible at the cost of increased hardware. On the other hand, software redundancy is useful when the system cost is the restriction as no additional hardware is needed. 

\noindent Two types of replicated tasks are implemented:
\begin{itemize}
    \item 2oo2\footnote{MooN is read as M out of N. It shows how many valid outputs have to be present for valid operation e.g. 1oo2 means 1 valid output out of 2 have to be present for a valid operation} configuration or without recovery
    \item 2oo3 configuration or with recovery
\end{itemize}

Recovery of 2oo3 configuration can be achieved with voting logic. Voting logic can determine which two tasks have the same output and make it a valid one. The same is not possible with 2oo2 voting logic. \autoref{appendix_cmd_ref_freertos} contains the command reference for the replicated tasks.

\subsection{Architecture}

Replicated tasks can detect errors using at least two tasks performing identical operations. Tasks are independently processed by the processor. Output variables from tasks are compared in real-time. In case of a discrepancy in the output variables, an error callback is called where the user can process the error.

\autoref{fig:replicated_example} shows how replicated tasks with recovery work. It shows that all instances wait on the barrier. When all tasks have arrived their compare values are compared. In case of a mismatch, the callback given on task creation is called. 


\begin{figure}[H]

      \centering
      \includegraphics[width=\linewidth]{images/replicated_example.png}
      \caption{Replicated task with redundancy}
      \label{fig:replicated_example}
    
\end{figure}

Comparison logic is not a new task for itself. It is done within one of the tasks. Whichever arrives last. \code{vTaskDelete} function was modified so that when one of the replicated sub-tasks delete is requested all linked will be deleted. Tasks are linked over the TCBs\footnote{TCB - Task control block}.

\subsection{Limitiations}

The static creation of the timed tasks is not available.


\chapter{Secure bootloader}
\label{custom_bootloader}
\textbf{TODO} Koristeni coding standard.
\textbf{TODO} Kako izgleda flow bootloadera
\textbf{TODO} Opis vektora u cortex M-u
\textbf{TODO} Koja stanja ima
\textbf{TODO} Kako se updatea aplikacija
\textbf{TODO} Koje funkcije ima
\textbf{TODO} Persistent memory
\chapter{Demonstration of developed software}
\label{demonstration}

\todoi{Demonstration}
\include{chapters/conclusion}

\bibliography{../literatura}
\bibliographystyle{../fer}

\begin{abstract}

Functional safety and functional safety standards overview is given, along with an example of functional safety in embedded systems. Explanation of how ARM Cortex-R improves functional safety over Cortex M is provided. Implemented software for redundancy on Cortex M. The software includes a bootloader and a modified FreeRTOS operating system. The bootloader supports the ability to load new applications and contains a command shell with accompanying functions. The modified FreeRTOS operating system provides features of tracking the individual tasks execution time and task replication. The demonstration of the functionality is also contained.  % Gramarly OK

\keywords{Functional safety, IEC 61508,IEC 26262, bootloader, certification, ARM Cortex R, Cortex R, FreeRTOS, Cortex M, kernel, redudancy, lock-step} % Gramarly OK
\end{abstract}

\hrtitle{Programska i sklopovska arhitektura redundantnih ugradbenih računalnih sustava}
\begin{sazetak}

Opisana je funkcijska sigurnost i dan je pregled odgovarajućih standarda. Dodan je i primjer funkcijske sigurnosti u ugradbenim računalnim sustavima. Opisano je kako ARM Cortex R popravlja funkcijsku sigurnost u odnosu na ARM Cortex M.
Implementirana je programska potpora za ostvarivanje redundancije na ARM Cortex M mikrokontroleru. Programska potpora za redundanciju sadržava bootloader i modificirani operacijski sustav FreeRTOS. Bootloader podržava mogućnost učitavanja novih aplikacija i naredbenu ljusku s odgovarajućim funkcijama. Modificirani operacijski sustav FreeRTOS ima mogućnost praćenja vremena izvođenja pojedinih zadataka i sposobnost repliciranja zadataka (engl. task replication). Demonstracija funkcionalnosti je isto dostupna.

\kljucnerijeci{Funkcijska sigurnost, IEC 61508, IEC 26262, bootloader, certifikacija, ARM, Cortex R, FreeRTOS, Cortex M, jezgra, redundancija}
\end{sazetak}

\begin{appendices}
\chapter{FreeRTOS redudancy API command reference} % Grammarly OK
\label{appendix_cmd_ref_freertos}

Timed tasks:
\begin{itemize}

    \item \nameref{rt_cmd:xTaskCreateTimed}
    \item \nameref{rt_cmd:vTaskTimedReset}
    \item \nameref{rt_cmd:xTimerGetTaskHandle}
    
\end{itemize}

\noindent Replicated tasks:
\begin{itemize}

    \item \nameref{rt_cmd:xTaskCreateReplicated}
    \item \nameref{rt_cmd:xTaskSetCompareValue}
    \item \nameref{rt_cmd:vTaskSyncAndCompare}
    
\end{itemize}

\noindent General added functions:
\begin{itemize}

    \item \nameref{rt_cmd:eTaskGetType}
    \item \nameref{rt_cmd:xTimerPause}
    \item \nameref{rt_cmd:xTimerPauseFromISR}
    \item \nameref{rt_cmd:xTimerResume}
    \item \nameref{rt_cmd:xTimerResumeFromISR}
    \item \nameref{rt_cmd:xTimerIsTimerActiveFromISR}
    
\end{itemize}
\pagebreak
\section{xTaskCreateTimed -  Creates a timed task.}
\label{rt_cmd:xTaskCreateTimed}

\begin{minted}[breaklines, linenos]{c}

BaseType_t xTaskCreateTimed( TaskFunction_t pxTaskCode,
                    const char * const pcName,
                    const configSTACK_DEPTH_TYPE usStackDepth,
                    void * const pvParameters,
                    UBaseType_t uxPriority,
                    TaskHandle_t * const pxCreatedTask,
                    TickType_t xOverrunTime,
                    WorstTimeTimerCb_t pxOverrunTimerCb,
                    TickType_t xOverflowTime,
                    WorstTimeTimerCb_t pxOverflowTimerCb )
            
\end{minted}

\begin{lstlisting}        
Create a new timed task and add it to the list of tasks that are ready to
run.

Overrun timer is synchronous with the task and its counter is incremented
only when timed task is in running state. Overrun callback is called from
timer daemon. When timed task overruns it sends a signal to the timer daemon
and when callback is called is dependent on daemon's priority. If overrun
timer is not used send 0 for xOverrunTime or NULL for the callback.

Overflow timer is asynchronous with the task and its counter is incremented
every tick regardless of the state. Callback is called from timer daemon and
its punctuality is dependent on timer daemon's priority. If overflow timer
is not used send 0 for xOverflowTime or NULL for the callback.

Internally, within the FreeRTOS implementation, tasks use two blocks of
memory.  The first block is used to hold the task's data structures.  The
second block is used by the task as its stack.  If a task is created using
xTaskCreateTimed() then both blocks of memory are automatically dynamically
allocated inside the xTaskCreate() function.  (see
http://www.freertos.org/a00111.html). Static version of the function is not
implemented.

Input paramters:
 - pvTaskCode - Pointer to the task entry function.  Tasks
must be implemented to never return (i.e. continuous loop).

- pcName - A descriptive name for the task.  This is mainly used to
facilitate debugging.  Max length defined by configMAX_TASK_NAME_LEN - default is 16.

- usStackDepth -The size of the task stack specified as the number of
variables the stack can hold - not the number of bytes.  For example, if
the stack is 16 bits wide and usStackDepth is defined as 100, 200 bytes
will be allocated for stack storage.

- pvParameters - Pointer that will be used as the parameter for the task
being created.

- uxPriority - The priority at which the task should run.  Systems that
include MPU support can optionally create tasks in a privileged (system)
mode by setting bit portPRIVILEGE_BIT of the priority parameter.  For
example, to create a privileged task at priority 2 the uxPriority parameter
should be set to ( 2 | portPRIVILEGE_BIT ).

- pvCreatedTask - Used to pass back a handle by which the created task
can be referenced.

- xOverrunTime - Runtime of the task after which callback will be called.

- pxOverrunTimerCb - Pointer to the function that will be called if task
runs longer than xOverrunTime without reseting the timed task. Overrun timer
is synchronous with the task and its tick is only incremented when timed
task is in running state.

- xOverflowTime - Asynchronous timer time. After xOverflowTime
pxOverflowTimerCb will be called.

- pxOverflowTimerCb - Pointer to the function that will be called after
xOverflowTime. Overflow timer is asynchronous from the task and its value is
incremented every tick.

Returns pdPASS if the task was successfully created and added to a ready
list, otherwise an error code defined in the file projdefs.h
Example usage:

\end{lstlisting}

\begin{minted}[breaklines, linenos]{c}
 // Task to be created.
 void vTaskTimedCode( void * pvParameters )
 {
     for( ;; )
     {
         // Task code goes here.

         // Reset the timer.
         vTaskTimedReset(NULL);
     }
 }

// Function to be called if timer overflows.
void vTaskOverflowCallback ( WorstTimeTimerHandle_t xTimer )
{
    // Timeout callback code.

    // Maybe task deletion is needed. Calling vTaskDelete automatically deletes
    // the timer too. Do NOT delete the timer directly. That will cause
    // undefined behavior when deleting the task.
    vTaskDelete( xTimerGetTaskHandle( xTimer ) );
}
// Function to be called if timer overflows.
void vTaskOverrunCallback ( WorstTimeTimerHandle_t xTimer )
{

    // Timeout callback code.

    // Maybe task deletion is needed. Calling vTaskDelete automatically deletes
    // the timer too. Do NOT delete the timer directly. That will cause
    // undefined behavior when deleting the task.
    vTaskDelete( xTimerGetTaskHandle( xTimer ) );
}

 // Function that creates a task.
 void vOtherFunction( void )
 {
 static uint8_t ucParameterToPass;
 TaskHandle_t xHandle = NULL;

     // Create the task, storing the handle.  Note that the passed parameter ucParameterToPass
     // must exist for the lifetime of the task, so in this case is declared static.  If it was just an
     // an automatic stack variable it might no longer exist, or at least have been corrupted, by the time
     // the new task attempts to access it.
     xTaskCreate( vTaskCode,
                  "NAME",
                  STACK_SIZE,
                  &ucParameterToPass,
                  tskIDLE_PRIORITY,
                  &xHandle,
                  pdMS_TO_TICKS(1 * 1000),
                  vTaskOverrunCallback,
                  pdMS_TO_TICKS(2 * 1000),
                  vTaskOverflowCallback );
     configASSERT( xHandle );

     // Use the handle to delete the task.
     if( xHandle != NULL )
     {
         vTaskDelete( xHandle );
     }
 }

\end{minted}

\section{vTaskTimedReset -  Resets the timer of timed task.}
\label{rt_cmd:vTaskTimedReset}


\begin{minted}[breaklines, linenos]{c}
void vTaskTimedReset( TaskHandle_t pxTaskHandle )
\end{minted}

\begin{lstlisting}
Reset the timer of the timed task.

- Warning - Shall only be used for timed tasks.

Input parameters:

- pxTaskHandle - Handle of the task whose timer shall be reset.
Passing a NULL handle results in reseting the timer of the calling task.

Example usage:
\end{lstlisting}

\begin{minted}[breaklines, linenos]{c}
void vTimedTask( void * pvParameters )
{
    for( ;; )
    {
        // Task code goes here.

        vTaskTimedReset(NULL);
    }
}
\end{minted}

\section{xTimerGetTaskHandle -  Gets the corresponding timed task handle from the timer handle.}
\label{rt_cmd:xTimerGetTaskHandle}
\begin{minted}[breaklines, linenos]{c}
TaskHandle_t xTimerGetTaskHandle( const TimerHandle_t xTimer )
\end{minted}
\begin{lstlisting}
Returns the timed task handle assigned to the timer. Task handle is an union
with timer ID and that is why they are mutually exclusive.

Task handle is assigned to the timer when creating the timed task.

WARNING: Setting the timer ID also sets the task handle. Changing the timer
ID can lead to undefined behavior.

Input parameters:

- xTimer - The timer being queried.

Example usage:

- See xTaskCreateTimed

\end{lstlisting}
\section{xTaskCreateReplicated -  Creates a replicated task.}
\label{rt_cmd:xTaskCreateReplicated}

\begin{minted}[breaklines, linenos]{c}
BaseType_t xTaskCreateReplicated( TaskFunction_t pxTaskCode,
                                  const char * const pcName,
                                  const configSTACK_DEPTH_TYPE usStackDepth,
                                  void * const pvParameters,
                                  UBaseType_t uxPriority,
                                  TaskHandle_t * const pxCreatedTask,
                                  uint8_t ucReplicatedType,
                                  RedundantValueErrorCb_t pxRedundantValueErrorCb )
\end{minted}

\begin{lstlisting}
Create a new replicated task and add it to the list of tasks that are ready
to run. Replicated task is used to achieve redundancy of the software at the
expense of slower execution. Task executes slower because it is replicated
two or three times. Depending  on the type chosen. On every call to
vTaskSyncAndCompare task is suspended until every replicated task arrives to
the same point. When every task is in the synchronization function
comparison is done. If any of the comparison results differ callback
function pxRedundantValueErrorCb is called. In the callback function
user can access the compare values and choose whether to delete all the
tasks.

Internally, within the FreeRTOS implementation, tasks use two blocks of
memory.  The first block is used to hold the task's data structures.  The
second block is used by the task as its stack.  If a task is created using
xTaskCreateReplicated() then both blocks of memory are automatically
dynamically allocated inside the xTaskCreateReplicated() function.  (see
http://www.freertos.org/a00111.html). Static version of this function is not
implemented.

Input parameters:
- pvTaskCode - Pointer to the task entry function.  Tasks
must be implemented to never return (i.e. continuous loop).

- pcName - A descriptive name for the task.  This is mainly used to
 facilitate debugging.  Max length defined by configMAX_TASK_NAME_LEN - default
 is 16.

- usStackDepth - The size of the task stack specified as the number of
variables the stack can hold - not the number of bytes.  For example, if
the stack is 16 bits wide and usStackDepth is defined as 100, 200 bytes
will be allocated for stack storage.

- pvParameters - Pointer that will be used as the parameter for the task
being created.

- uxPriority - The priority at which the task should run.  Systems that
include MPU support can optionally create tasks in a privileged (system)
mode by setting bit portPRIVILEGE_BIT of the priority parameter.  For
example, to create a privileged task at priority 2 the uxPriority parameter
should be set to ( 2 | portPRIVILEGE_BIT ).

- pvCreatedTask - Used to pass back a handle by which the created task
can be referenced.

- ucReplicatedType - Valid values: taskREPLICATED_NO_RECOVERY and
taskREPLICATED_RECOVERY. No recovery is faster as it created only two
instances, but recovery is not possible. Recovery creates three identical
tasks. Recovery is possible with 2 out of 3 logic.

- pxRedundantValueErrorCb - Function to be called when compare values do
not match. Return value determines whether calling redundant task will be
deleted.

Returns pdPASS if the task was successfully created and added to a ready
list, otherwise an error code defined in the file projdefs.h


Example usage:
\end{lstlisting}
\begin{minted}[breaklines, linenos]{c}
// Task to be created.
void vTaskCode( void * pvParameters )
{
    for( ;; )
    {
        // Task code goes here.

        vTaskSyncAndCompare(&xCompareValue);
    }
}

// NOTE: This function is called from the redundant task and not daemon.
uint8_t ucCompareErrorCb (CompareValue_t * pxCompareValues, uint8_t ucLen)
{
    // Iterate through compare values.
    for(uint8_t iii = 0; iii < ucLen; i++)
    {
        pxCompareValue[iii]
        .
        .
        .
    }

    return pdTRUE; // Signaling to delete the redundant task.
}

// Function that creates a task.
void vOtherFunction( void )
{
static uint8_t ucParameterToPass;
TaskHandle_t xHandle = NULL;

    // Create the task, storing the handle.  Note that the passed parameter ucParameterToPass
    // must exist for the lifetime of the task, so in this case is declared static.  If it was just an
    // an automatic stack variable it might no longer exist, or at least have been corrupted, by the time
    // the new task attempts to access it.
    xTaskCreateReplicated( vTaskCode, "NAME", STACK_SIZE, &ucParameterToPass, tskIDLE_PRIORITY, &xHandle, taskREPLICATED_RECOVERY, ucCompareErrorCb );
    configASSERT( xHandle );

    // Use the handle to delete the task.
    if( xHandle != NULL )
    {
        vTaskDelete( xHandle );
    }
}

\end{minted}

\section{xTaskSetCompareValue -  Sets a compare value for the calling task.}
\label{rt_cmd:xTaskSetCompareValue}

\begin{minted}[breaklines, linenos]{c}
void xTaskSetCompareValue( CompareValue_t xNewCompareValue )

\end{minted}
\begin{lstlisting}
Sets the compare value. Compare value is used with replicated tasks. They
are used in vTaskSyncAndCompare function for figuring if there is a
difference between the tied task executions.
Input parameters:
- xNewCompareValue - New compare value to set.

\end{lstlisting}
\section{vTaskSyncAndCompare -  Syncronizes the replicated tasks and compares compare values.}
\label{rt_cmd:vTaskSyncAndCompare}

\begin{minted}[breaklines, linenos]{c}
void vTaskSyncAndCompare( const CompareValue_t * const pxNewCompareValue )


\end{minted}
\begin{lstlisting}
Waits until every replicated task is finished. When every task is finished
function compares the compare values and if there is a mismatch it calls the
predefined callback.

- Warning - Shall only be used for replicated tasks.

Input parameters:

- pxNewCompareValue - Pointer of the compare value to be copied from. If
NULL is passed in, previous compare value is used.

Example usage:
\end{lstlisting}
\begin{minted}[breaklines, linenos]{c}
void vReplicatedTask( void * pvParameters )
{
    for( ;; )
    {
        // Task code goes here.

        vTaskSyncAndCompare(&xCompareValue);
    }
}

\end{minted}

\section{eTaskGetType -  Get the type of the task.}
\label{rt_cmd:eTaskGetType}
\begin{minted}[breaklines, linenos]{c}
eTaskType eTaskGetType( TaskHandle_t pxTaskHandle )
\end{minted}
\begin{lstlisting}
Get the type of the task.

Input parameters:

- pxTaskHandle - Handle of the task to be queried.  Passing a NULL
handle results in getting the type of calling task.

\end{lstlisting}

\section{xTimerPause -  Pauses the timer.}
\label{rt_cmd:xTimerPause}

\begin{minted}[breaklines, linenos]{c}
BaseType_t xTimerPause( TimerHandle_t xTimer, TickType_t xTicksToWait )
\end{minted}

\begin{lstlisting}
Timer functionality is provided by a timer service/daemon task.  Many of the
public FreeRTOS timer API functions send commands to the timer service task
through a queue called the timer command queue.  The timer command queue is
private to the kernel itself and is not directly accessible to application
code.  The length of the timer command queue is set by the
configTIMER_QUEUE_LENGTH configuration constant.

xTimerPause() pauses a timer. If timer was not running before it is ignored.
Pausing remembers how many ticks until the deadline are needed and on next
xTimerResume() timer will trigger only after the ticks set by pause.

Pausing assures timer is in stopped state.
 
- xTimer - The handle of the timer being paused.

- TicksToWait - Specifies the time, in ticks, that the calling task should
be held in the Blocked state to wait for the stop command to be successfully
sent to the timer command queue, should the queue already be full when
xTimerPause() was called.  xTicksToWait is ignored if xTimerPause() is called
before the scheduler is started.

Returns pdFAIL if the pause command could not be sent to  timer command queue
even after xTicksToWait ticks had passed.  pdPASS will be returned if the
command was successfully sent to the timer command queue.
When the command is actually processed will depend on the priority of the
timer service/daemon task relative to other tasks in the system.  The timer
service/daemon task priority is set by the configTIMER_TASK_PRIORITY
configuration constant.

\end{lstlisting}

\section{xTimerPauseFromISR -  Pauses the timer from interrupt service routine.}
\label{rt_cmd:xTimerPauseFromISR}
\begin{minted}[breaklines, linenos]{c}
 BaseType_t xTimerPauseFromISR( TimerHandle_t xTimer,
                                BaseType_t *pxHigherPriorityTaskWoken );
\end{minted}

\begin{lstlisting}

 A version of xTimerPause() that can be called from an interrupt service
 routine.

- xTimer  - The handle of the timer being paused.

- pxHigherPriorityTaskWoken - The timer service/daemon task spends most
of its time in the Blocked state, waiting for messages to arrive on the
timer command queue.  Calling xTimerPauseFromISR() writes a message to
the timer command queue, so has the potential to transition the timer
service/daemon task out of the Blocked state.  If calling
xTimerPauseFromISR() causes the timer service/daemon task to leave the
Blocked state, and the timer service/ daemon task has a priority equal
to or greater than the currently executing task (the task that was
interrupted), then *pxHigherPriorityTaskWoken will get set to pdTRUE
internally within the xTimerPauseFromISR() function.  If xTimerPauseFromISR()
sets this value to pdTRUE then a context switch should be performed before
the interrupt exits.

Returns pdFAIL if the pause command could not be sent to  the timer
command queue.  pdPASS will be returned if the command was successfully
sent to the timer command queue.  When the command is actually processed
will depend on the priority of the timer service/daemon task relative
to other tasks in the system.  The timer service/daemon task priority is
set by the configTIMER_TASK_PRIORITY configuration constant.

\end{lstlisting}
\section*{xTimerResume -  Resumes the timer.}
\label{rt_cmd:xTimerResume}
\begin{minted}[breaklines, linenos]{c}
 BaseType_t xTimerResume( TimerHandle_t xTimer, TickType_t xTicksToWait )
\end{minted}
\begin{lstlisting}
Timer functionality is provided by a timer service/daemon task.  Many of the
public FreeRTOS timer API functions send commands to the timer service task
through a queue called the timer command queue.  The timer command queue is
private to the kernel itself and is not directly accessible to application
code.  The length of the timer command queue is set by the
configTIMER_QUEUE_LENGTH configuration constant.

xTimerResume() resumes a timer. If timer was not running before it acts as
xTimerStart. If timer saw stopped prior to the call with xTimerPause than it
places a deadline in daemon task from the time timer left of and not the
full period.

Resuming assures timer is in running state. If the timer is not stopped,
deleted, or reset in the mean time, the callback function associated with the
timer will get called 'n' ticks after xTimerStart() was called, where 'n' is
the time left from when last pause was called.

- xTimer - The handle of the timer being resumed.

- xTicksToWait - Specifies the time, in ticks, that the calling task should
be held in the Blocked state to wait for the resume command to be
successfully sent to the timer command queue, should the queue already be
full when xTimerResume() was called.  xTicksToWait is ignored if
xTimerResume() is called before the scheduler is started.

pdFAIL will be returned if the resume command could not be sent to
the timer command queue even after xTicksToWait ticks had passed.  pdPASS
will be returned if the command was successfully sent to the timer command
queue. When the command is actually processed will depend on the priority
of the timer service/daemon task relative to other tasks in the system.
The timer service/daemon task priority is set by the
configTIMER_TASK_PRIORITY configuration constant.
\end{lstlisting}

\section{xTimerResumeFromISR -  Resumes the timer from interrupt service routine.}
\label{rt_cmd:xTimerResumeFromISR}
\begin{minted}[breaklines, linenos]{c}
 BaseType_t xTimerResumeFromISR(  TimerHandle_t xTimer,
                                  BaseType_t *pxHigherPriorityTaskWoken )
\end{minted}

\begin{lstlisting}
A version of xTimerResume() that can be called from an interrupt service
routine.

- xTimer - The handle of the timer being resumed.

- pxHigherPriorityTaskWoken - The timer service/daemon task spends most
of its time in the Blocked state, waiting for messages to arrive on the
timer command queue.  Calling xTimerPauseFromISR() writes a message to
the timer command queue, so has the potential to transition the timer
service/daemon task out of the Blocked state.  If calling
xTimerPauseFromISR() causes the timer service/daemon task to leave the
Blocked state, and the timer service/ daemon task has a priority equal
to or greater than the currently executing task (the task that was
interrupted), then *pxHigherPriorityTaskWoken will get set to pdTRUE
internally within the xTimerPauseFromISR() function.  If xTimerPauseFromISR()
sets this value to pdTRUE then a context switch should be performed before
the interrupt exits.

pdFAIL will be returned if the resume command could not be sent to
the timer command queue.  pdPASS will be returned if the command was
successfully sent to the timer command queue.  When the command is actually
processed will depend on the priority of the timer service/daemon task
relative to other tasks in the system.  The timer service/daemon task
priority is set by the configTIMER_TASK_PRIORITY configuration constant.

\end{lstlisting}
\section{xTimerIsTimerActiveFromISR -  Checks if timer is active from interrupt service routine.}
\label{rt_cmd:xTimerIsTimerActiveFromISR}

\begin{minted}[breaklines, linenos]{c}
BaseType_t xTimerIsTimerActiveFromISR( TimerHandle_t xTimer );
\end{minted}
\begin{lstlisting}
A version of xTimerIsTimerActive() that can be called from an interrupt service
routine.

- xTimer - The handle of the timer that is to be checked.

pdFAIL will be returned if the reset command could not be sent to
the timer command queue.  pdPASS will be returned if the command was
successfully sent to the timer command queue.  When the command is actually
processed will depend on the priority of the timer service/daemon task
relative to other tasks in the system, although the timers expiry time is
relative to when xTimerResetFromISR() is actually called.  The timer service/daemon
task priority is set by the configTIMER_TASK_PRIORITY configuration constant.

\end{lstlisting}

\chapter{Developed bootloader command reference} % Grammarly OK
\label{appendix_cmd_ref_bootloader}

Important notices:

\begin{itemize}
 \item Every execute of a command must end with \textbackslash r \textbackslash n

 \item Commands are case insensitive
 
 \item On error bootloader returns "ERROR:<Explanation of error>"
 
 \item Optional parameters are surrounded with [] e.g. [example]

\end{itemize}

\noindent List of all commands:

\begin{itemize}

    \item \nameref{bl_cmd:version}
    \item \nameref{bl_cmd:help}
    \item \nameref{bl_cmd:reset}
    \item \nameref{bl_cmd:cid}
    \item \nameref{bl_cmd:get-rdp-level}
    \item \nameref{bl_cmd:jump-to}
    \item \nameref{bl_cmd:flash-erase}
    \item \nameref{bl_cmd:flash-write}
    \item \nameref{bl_cmd:mem-read}
    \item \nameref{bl_cmd:update-act} 
    \item \nameref{bl_cmd:update-new}
    \item \nameref{bl_cmd:en-write-prot}
    \item \nameref{bl_cmd:dis-write-prot}
    \item \nameref{bl_cmd:get-write-prot}
    \item \nameref{bl_cmd:exit}

\end{itemize}



\section{version - Gets a version of the bootloader.}
\label{bl_cmd:version}

Parameters:
\begin{lstlisting}
 - None
Execute command: 

    > version  
    
Response: 

    v1.0  
\end{lstlisting}
    
    
\section{help - Makes life easier.}
\label{bl_cmd:help}

\begin{lstlisting}
Parameters:

 -  None

Execute command: 

    > help  
    
Response: 

    <List of all available commands and examples>
\end{lstlisting}

\section{reset - Resets the microcontroller.}
\label{bl_cmd:reset}

\begin{lstlisting}
Parameters:

 - None

Execute command: 

    > reset
    
Response: 

    OK
\end{lstlisting}


\section{cid - Gets chip identification number.}
\label{bl_cmd:cid}

\begin{lstlisting}
Parameters:

 - None

Execute command: 

    > cid  
    
Response: 

    0x413
\end{lstlisting}

\section{get-rdp-level - Gets read protection \texorpdfstring{\protect\cite[p.~93]{stm32f407_ref_man}}{}}
\label{bl_cmd:get-rdp-level}

\begin{lstlisting}
Parameters:

  - None

Execute command: 

    > get-rdp-level  
    
Response: 

    level 0
\end{lstlisting}
    
\section{jump-to - Jumps to a requested address.}
\label{bl_cmd:jump-to}

\begin{lstlisting}
Parameters:

- addr - Address to jump to in hex format (e.g. 0x12345678), 0x can be omitted

Execute command: 

    > jump-to addr=0x87654321 
     
Response: 

    OK
\end{lstlisting}
     
\section{flash-erase - Erases flash memory.}
\label{bl_cmd:flash-erase}

\begin{lstlisting}
Parameters:

- type - Defines type of flash erase. "mass" erases all sectors, "sector" erases only selected sectors
    
- sector - First sector to erase. Bootloader is on sectors 0, 1 and 2. Not needed with mass erase
    
- count - Number of sectors to erase. Not needed with mass erase

Execute command: 

    > flash-erase sector=3 type=sector count=4 
     
Response: 

    OK
\end{lstlisting}

\section{flash-write - Writes to flash memory.}
\label{bl_cmd:flash-write}

\begin{lstlisting}
Parameters:

 - start - Starting address in hex format (e.g. 0x12345678), 0x can be omitted
     
 - count - Number of bytes to write, without checksum. Chunk size: 5120
 
 - [cksum] - Defines the checksum to use. If not present no checksum is assumed. WARNING: Even if checksum is wrong data will be written into flash memory!
 
      - "sha256" - Gives best protection (32 bytes), slowest, uses software implementation
           
      - "crc32" - Medium protection (4 bytes), fast, uses hardware implementation.

      - "no" - No protection, fastest

Note:

  When using crc-32 checksum sent data has to be divisible by 4

Execute command: 

    > flash-write start=0x87654321 count=64 cksum=crc32  
    
Response: 

    chunks:1

    chunk:0|length:64|address:0x87654321

    ready
    
Send bytes:

    <64 bytes>
    
Response:

    chunk OK

    checksum|length:4

    ready
 
Send checksum:
     
     <4 bytes>
     
Response:
 
    OK
\end{lstlisting}

\section{mem-read - Read bytes from memory.}
\label{bl_cmd:mem-read}

\begin{lstlisting}
Parameters:

- start - Starting address in hex format (e.g. 0x12345678), 0x can be omitted
     
- count - Number of bytes to read

Execute command: 

    > mem-read start=0x87654321 count=3  
    
Response: 

    <3 bytes starting from the address 0x87654321>
    
Note:
- Entering invalid read address crashes the program and reboot is required. 
\end{lstlisting}

\section{update-act - Updates active application from new application memory area.}
\label{bl_cmd:update-act}

\begin{lstlisting}
Parameters:
- [force] - Forces update even if not needed

   - "true" - Force the update
                
   - "false" - Don't force the update


Execute command: 

    > update-act force=true
    
Response: 

    No update needed for user application
    Updating user application
    OK
\end{lstlisting}
    

\section{update-new - Updates new application.}
\label{bl_cmd:update-new}

\begin{lstlisting}
Parameters:

 - count - Number of bytes to write, without checksum. Chunk size: 5120
 
 - type - Type of application coding
       
      - "bin" - Binary format (.bin)
                
      - "hex" - Intel hex format (.hex)
      
      - "srec" - Motorola S-record format (.srec)
 
 - [cksum] - Defines the checksum to use. If not present no checksum is assumed. WARNING: Even if checksum is wrong data will be written into flash memory!
 
      - "sha256" - Gives best protection (32 bytes), slowest, uses software implementation
           
      - "crc32" - Medium protection (4 bytes), fast, uses hardware implementation.

      - "no" - No protection, fastest


Execute command: 

    > update-new count=4 type=bin cksum=sha256
    
Response: 

    chunks:1

    chunk:0|length:4|address:0x08080000

    ready
    
Send bytes:

    <4 bytes>
    
Response:

    chunk OK

    checksum|length:32

    ready
 
Send checksum:
     
     <32 bytes>
     
Response:
 
    OK
\end{lstlisting}

\section{en-write-prot - Enables write protection per sector.}
\label{bl_cmd:en-write-prot}

\begin{lstlisting}
Parameters:

- mask - Mask in hex form for sectors where LSB corresponds to sector 0

Execute command: 

    > en-write-prot mask=0xFF0
    
Response: 

    OK
\end{lstlisting}
    
\section{dis-write-prot - Disables write protection per sector.}
\label{bl_cmd:dis-write-prot}

\begin{lstlisting}
Parameters:

- mask - Mask in hex form for sectors where LSB corresponds to sector 0

Execute command: 

    > dis-write-prot mask=0xFF0
    
Response: 

    OK
\end{lstlisting}

\section{get-write-prot - Returns bit array of sector write protection.}
\label{bl_cmd:get-write-prot}Parameters:

\begin{lstlisting}
- None

Execute command: 

    > get-write-prot
    
Response: 

    0b100000000010
\end{lstlisting}

\section{exit - Exits the bootloader and starts the user application.}
\label{bl_cmd:exit}

\begin{lstlisting}
Parameters:

- None

Execute command: 

    > exit  
    
Response: 

    Exiting
\end{lstlisting}


\end{appendices}

\end{document}
